\section{Proposed Design}
\label{section:propeseddesign}
To develop a climate and activity based outfit suggestion tool, I will need to break the problem into parts.
At it's core will be a Machine Learning algorithm along with a datastore for the training data. Wrapping that will
be a query system and optimistically a mobile app for training the system. Additionally, I will need to determine 
the best way to represent the problem set in terms of features. Here, I break down the parts of the system and what
work I envision is needed.

\subsection{Learning Core}
Firstly, I will need to determine the best Machine Learning method to use. I have found a number of articles evaluating
various techniques in other applications. They should give me insight into which technique is most appropriate for Gear
Smarts. Some options include:

\begin{description}
  \item{kNN:} These can help to simplify a problem when you don't have a good model of the larger application
  \item{Decision tree:} Again, this is a relatively simple implementation that replaces modeling with basic relationships
  \item{Bayesian network:} Takes a statistical approach to modelling data
  \item{Support Vector Machine:} Non-linear classification using the ''kernel trick``. A short-coming is SVMs must classify
  datapoints into one of two categories. While this can be worked with for Gear Smarts, it has the potential to be limiting.
  \item{Artificial Neural Network:} These can adapt to a huge variety of problems, and are interesting in that they emulate
  human (and animal) learning. A major down-side is they can be complex and costly to implement.
  \item{Others:} There are many many other machine learning techniques that I could also investigate.
\end{description}

