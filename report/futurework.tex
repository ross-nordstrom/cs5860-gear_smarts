\section{Future Work}
\label{section:futurework}
As the vision of GearSmarts was described in Section \ref{section:design}, the main missing components are simply a UI
and searching for an optimal outfit out of a given wardrobe. Additionally there is room to improve the classifier and
understanding of the outfit comfort problem modeling.

\subsection{User Interface}
\label{section:ui}
While it was appropriate to simply use scripts for the initial evaluation and implementation of the GearSmarts API for
this report, eventual productization of the tool will require a UI with which users can easily use the API. The vision
for the UI is that it will store and model a user's wardrobe (their collection of known articles of clothing), against
which it can search combinations for the optimal outfit for a given activity in queried weather for the desired location
and date. Additionally, the UI should make it easy and enjoyable for users to log their outfits and comfort level in order
to train the system. The best approach is likely to gameify it somehow so that users are rewarded for recording more
data, thus better training GearSmarts.

\subsection{Modeling and Configuration}
\label{section:modelingandconfig}
As this application of machine learning is exploratory, the approach taken for modeling the problem space, the
SVM configuration used, and the weather data selected in GearSmarts is expected to evolve as more data is collected and
a better understanding of the problem space is acquired. Additonal steps and layers may be beneficial to the system, such
as clustering users into groups of similar heat-tolerances. Then each of these clusters could be trained and classified in
their own namespace in GearSmarts, thus isolating the ``run-hot'' from the ``always-cold'' users.