\section{Proposed Design}
\label{section:propeseddesign}
To develop a climate and activity based outfit suggestion tool, the problem will be broken into parts.
At its core will be a machine learning algorithm along with a datastore for the training data. Wrapping that will
be a query system and optimistically a mobile app for training the system. Additionally, a feature-based representation
will need to be determined. In this section, the envisioned parts of the system are presented.

\subsection{Learning Core}
Firstly, a machine learning method must be selected. Existing articles should give insight into which technique is
most appropriate for Gear Smarts. Some options are listed below:

\begin{description}
  \item{kNN:} Can help to simplify a problem in the absence of a good model of the larger application
  \item{Decision tree:} Again, this is a relatively simple implementation that replaces modeling with basic relationships
  \item{Bayesian network:} Takes a statistical approach to modeling data
  \item{Support Vector Machine:} Non-linear classification using the ``kernel trick.'' A short-coming is solving for multiple
  classes can add complexity, as a common technique is to use multiple binary SVM classifiers. While this can be worked with
  for Gear Smarts, it has the potential to be limiting.
  \item{Artificial Neural Network:} These can adapt to a huge variety of problems, and are interesting in that they emulate
  human (and animal) learning. A major down-side is they can be complex and costly to implement.
  \item{Others:} There are many many other machine learning techniques that could also be investigated.
\end{description}

Regardless of the algorithm chosen, a training dataset will need to be constructed with which to prepare the system. To do this,
data will be synthesized from both real-world experience and common sense.

\subsection{Problem Modeling}
Before using a machine learning system, the problem must be modeled properly. It can be broken down into three fundamental parts,
described below.

\begin{description}
  \item{Outfit:} Outfits can either be modeled individually or as a summation of individual articles (potentially introducing
  a knapsack problem \footnote{The knapsack problem seeks to optimize a value by searching combinations of a set. In Gear Smarts,
  we might represent an outfit with a KPI that should closely match a KPI produced by a function of the weather. Meanwhile, each
  article of clothing in a wardrobe could have a KPI suggesting how warm it is. To properly match the knapsack problem, we might
  introduce a mobility metric for each article.}). Additionally, they can be represented categorically (e.g. vest or full-zip and
  fleece, down, and cotton) or by derived characteristics (e.g. thermal retention, mobility, wind-resistance).
  \item{Weather:} Weather should likely be characterized in terms of temperature, humidity, wind, pressure,
  cloud cover, and precipitation.
  \item{Comfort:} Intuitively, there are three basic comfort levels relevant to Gear Smart: too cold, comfortable, and too hot.
\end{description}

\subsection{User Interface}
A simple way to enable user interaction would be to expose the machine learning system via a RESTful API and consume it
through a mobile app. The interface on the app would simply ask the user what their outfit was that day, and how comfortable
they were. Under the hood, the app would record the user's geolocation and query some service for weather information
about that location on that day. Then it would convert that data into a representative feature set with which to send
the machine learning API for training.

Later, the user could use the app to ask the question ``what should I wear today?'' Using the trained data, the app can
suggest which outfit to wear. Something to determine is whether or not to burden the user with managing
their wardrobe in the system, allowing the system to more usefully suggest outfits. The alternative is to hope the
learning mechanism would adapt to which items each user has based on their feedback.
