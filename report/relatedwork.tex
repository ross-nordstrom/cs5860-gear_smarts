\section{Related Work}
\label{section:relatedwork}
While there have been a number of works in automating outfit selection, they are all geared towards
fashion selection either for aesthetics or occasions (e.g. wedding, funeral, interview, or church).
There does not seem to be any existing work in the climate-outfitting field, let alone with an
activity focus.

In general, related works fall into three categories: fashionable outfit selection, category-based
outfit selection, and general temperature/climate automation. Here several related works are briefly introduced:

\subsection{Fashionable Outfit Selection}
Yu et al. \cite{Dressup} uses a wardrobe of known clothes to suggest outfits based on the user's appearance.
Outfits are suggested based on a dress code constraint, which is a similar task to
the challenge of activity-based outfit selection.

Rode et al. \cite{SmartCloset} works on outfit selection and sharing, focused on comparing and lending wardrobe
to others. There may be some similarities between this and the idea of comparing user outfit preferences
for GearSmarts.

Liu et al. \cite{MagicCloset} use Amazon's Mechanical Turk to fill a database of available-to-purchase outfits.
Using this dataset, they constructed an occasion-oriented outfit suggestion tool similar to Yu et al.\cite{Dressup}

Clear et al. \cite{ThermalComfort} discusses a new approach to thermal control of homes. This could relate well to
adapting GearSmarts to sub-activities with different thermal properties (e.g. mogul vs groomer skiing).

\subsection{Machine Learning Technique Selection}
There is extensive research discussing and comparing the various options for machine learning techniques.
Some relevant papers used in the decision of which technique to use for GearSmarts include \cite{ML:MapReduceClusters},
Clear et al. \cite{ML:ManufacturingSystems}, \cite{ML:IPTraffic}, and \cite{ML:GeoMapping}. Ultimately, SVMs were selected
for their simplicity to use thanks to widely available libraries, support for multi-class classification, and configurability
via various kernels. Regardless of SVMs being the choice as of this report, the design of GearSmarts allows for SVMs
to easily be swapped out for other machine learning techniques, while affecting very little of the overall system and
architecture. A brief overview of some considered techniques are described below.

\begin{description}
  \item{Support Vector Machine \cite{SVM}:} Non-linear classification using the ``kernel trick.'' A short-coming is solving for multiple
  classes adds complexity, solved by the use of multiple binary SVM classifiers.
  \item{kNN \cite{KNN}:} Can help to simplify a problem in the absence of a good model of the larger application
  \item{Decision tree \cite{DecisionTree}:} Again, this is a relatively simple implementation that replaces modeling with basic relationships
  \item{Bayesian network \cite{BayesianNetwork}:} Takes a statistical approach to modeling data
  \item{Artificial Neural Network \cite{ANN}:} These can be applied to a huge variety of problems, and are interesting in that they emulate
  human (and animal) learning. A major downfall is they can be complex and costly to implement.
  \item{Others:} There are countless other machine learning techniques that could also be investigated.
\end{description}