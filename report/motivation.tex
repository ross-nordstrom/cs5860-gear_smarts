\section{Motivation}
\label{section:motivation}
This research seeks to implement a tool or application to assist in the selection of clothing based on
historical data, an activity, and a locale (or weather). The purpose of this project is to use a real-world
problem to drive the use of machine learning in an integrated system. Originally, the project idea came from trying
to decide which layers to wear by recalling what was used before in similar weather conditions. This is a
useful and intersting application of machine learning because it has real-world value and is extensible to a variety
of areas, including a number of sports or activites, or what to pack for a vacation.

\subsection{Value}
Many outdoor activities, such as skiing, require planning ahead for the entire day. An especially important
decision is what to wear, or more specifically what layers to wear (e.g. a base layer, a fleece jacket, and a shell).
This decision is strongly dependent on several variables, including the temperature throughout the day, the weather
quality (sunny, cloudy, snowing), and even which specific activity will be done (such as mogul runs or groomers).
Additionally, a person's ``mental repository'' indexing what they've worn in the past for similar weather conditions
may be sparse due to infrequently taking part in the activity, or conducting the activity in a new environment (e.g.
after moving).

If there were a tool to help apply a person's own history as well as that of similar users to a given day's weather
forecast, the decision could become easy.

\subsection{Extensibility}
This use of machine learning is applicable to anyone venturing into variable climates and/or activities.
By quickly and simply categorizing
users and their current location, GearSmarts could develop an ever-growing database of (1) items to wear and their properties,
(2) user types (i.e. cold and heat tolerance), and (3) characteristics of various weather types (think windchill).

The approach taken in GearSmarts makes it a perfect candidate for machine learning. By namespacing
learned machines and taking generic feature vector plus classificaiton combinations, the API would enable consumers to
train on and predict any classification problem desired.

\subsection{Market}
\label{section:market}
In general, there has been an explosion in the popularity of apps and products surrounding sports and outdoor activities.
Of note is the EpicMix application and infrastructure \cite{EpicMix:Site}, which helps hundreds of thousands of skiers and snow
boarders \cite{EpicMix:PlayStore} gain insight into how much they are skiing by tracking and gamifying the lifts they ride.
Other industries, especially running and hiking, have also experienced a large growth in the use of technology to track and
augment user experience. With users becoming more accustomed to the use of technology to enhance their participation in outdoor
activities, there has never been a better time for a tool targeted at helping people decide how to gear up for the day.

Additionally, the knowledge base developed by this tool could be leveraged to help teach beginners the ideal dress in a sport or activity
based on what the experienced participants use. By introducing some leaderboard logic (``who has the most entries for a given activity'')
sample users could be identified to share their clothing decisions with inexperienced (few entries in that activity) users.

A final obvious use for this information and tool is to integrate it with the marketing for clothing and gear companies
involved in the activites recorded in GearSmarts. Knowing a user is about to, or planning to, engage in an activity, and
what outfit GearSmarts recommends would be a perfect opportunity to market particular brands selling items in the suggested outfit.