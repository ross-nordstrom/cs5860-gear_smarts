\section{Motivation}
\label{section:motivation}
This research seeks to implement a tool or application to assist in the selection of what to wear based on
historical data, an activity, and a locale (or weather). The purpose of this project is to use a real-world
problem to drive the use of machine learning in an integrated system. Originally, the project idea came from trying
to decide which layers to wear by recalling what was used before in similar weather conditions. This is a
useful and intersting application of machine learning because it has real-world value and is extensible to a variety
of areas, including a variety of sports or activites, or what to pack for a vacation.

\subsection{Value}
Many outdoor activities, such as skiing, require planning ahead for the entire day. An especially important
decision is what to wear, or more specifically what layers to wear (e.g. a base layer, a fleece jacket, and a shell).
This decision is strongly dependent on several variables, including the temperature throughout the day, the weather
quality (sunny, cloudy, snowing), and even which specific activity will be done (such as mogul runs or groomers).
Additionally, a person's ``mental repository'' indexing what they've worn in the past for similar weather conditions
may be sparse due to infrequently taking part in the activity, or conducting the activity in a new environment (e.g.
after moving).

If there were a tool to help apply a person's own history as well as that of similar users to a given day's weather
forecast, the decision could become easy.

\subsection{Extensibility}
This use of machine learning is applicable to anyone venturing into variable climates and/or activities.
By quickly and simply categorizing
users and their current locality, we could develop an ever-growing database of (1) items to wear and their properties,
(2) user types (i.e. cold/heat tolerance), and (3) characteristics of various weather types (think windchill).

\subsection{Market}
In general, there has been an explosion in the popularity of apps and products surrounding sports and outdoor activities.
Of note is the EpicMix application and infrastructure \cite{EpicMix:Site}, which helps hundreds of thousands of skiers and snow
boarders \cite{EpicMix:PlayStore} gain insight into how much they are skiing by tracking and gamifying the lifts they ride up.
Other industries, especially running and hiking, have experienced a large growth in the use of technology to augment and
track user experience. With users becoming more accustomed to the use of technology to augment their participation in outdoor
activities, there has never been a better time for a tool targeted at helping people decide how to gear up for the day.
